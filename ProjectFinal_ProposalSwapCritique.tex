%%%%%%%%%%%%%%%%%%%%%%%%%%%%%%%%%%%%%%%%%%%%%%%%%%%%%%%%%%%%%%%%%%%%%%%%%%%%%%%%%%%%%%%%%%%%%%%%
%
% CSCI 1430 Project Proposal Swap Critique Template
%
% This is a LaTeX document. LaTeX is a markup language for producing documents.
% Your task is to answer the questions by filling out this document, then to 
% compile this into a PDF document. 
% 
% TO COMPILE:
% > pdflatex thisfile.tex
%
% If you do not have LaTeX and need a LaTeX distribution:
% - Departmental machines have one installed.
% - Personal laptops (all common OS): http://www.latex-project.org/get/
%
% If you need help with LaTeX, come to office hours. Or, there is plenty of help online:
% https://en.wikibooks.org/wiki/LaTeX
%
% Good luck!
% James and the 1430 staff
%
%%%%%%%%%%%%%%%%%%%%%%%%%%%%%%%%%%%%%%%%%%%%%%%%%%%%%%%%%%%%%%%%%%%%%%%%%%%%%%%%%%%%%%%%%%%%%%%%
%
% How to include two graphics on the same line:
% 
% \includegraphics[width=0.49\linewidth]{yourgraphic1.png}
% \includegraphics[width=0.49\linewidth]{yourgraphic2.png}
%
% How to include equations:
%
% \begin{equation}
% y = mx+c
% \end{equation}
% 
%%%%%%%%%%%%%%%%%%%%%%%%%%%%%%%%%%%%%%%%%%%%%%%%%%%%%%%%%%%%%%%%%%%%%%%%%%%%%%%%%%%%%%%%%%%%%%%%

\documentclass[11pt]{article}

\usepackage[english]{babel}
\usepackage[utf8]{inputenc}
\usepackage[colorlinks = true,
            linkcolor = blue,
            urlcolor  = blue]{hyperref}
\usepackage[a4paper,margin=1.5in]{geometry}
\usepackage{stackengine,graphicx}
\usepackage{fancyhdr}
\setlength{\headheight}{15pt}
\usepackage{microtype}
\usepackage{times}
\usepackage{booktabs}

% From https://ctan.org/pkg/matlab-prettifier
\usepackage[numbered,framed]{matlab-prettifier}

\frenchspacing
\setlength{\parindent}{0cm} % Default is 15pt.
\setlength{\parskip}{0.3cm plus1mm minus1mm}

\pagestyle{fancy}
\fancyhf{}
\lhead{Final Project Proposal Swap Critique}
\rhead{CSCI 1430}
\rfoot{\thepage}

\date{}

\title{\vspace{-1cm}Final Project Proposal Swap Critique}

\begin{document}
\maketitle
\vspace{-1cm}
\thispagestyle{fancy}

\emph{Please make this document anonymous.}

\textbf{Team name: \emph{HERE PLEASE}}\\
\textbf{TA name: \emph{HERE PLEASE}}

\textbf{Critique for project of team name: \emph{HERE PLEASE}}

\emph{Note:} Please email your completed critique to your team's TA.

\section*{Critique Instructions}

You should have received an email from the STAs with an anonymous proposal from another team. 

Please read the other team's proposal, and analyze it to list at least 5 socially-responsible computing concerns. 

These could be direct responses to their answers to the SRC prompts, new aspects that you think the team missed, or something else. Each concern should be described in 1-2 sentences.

For each additional reasonable concern expressed beyond 5, you will earn +0.5 extra credit points on your final project grade [up to 5 points total].

\section*{Critique}

\begin{enumerate}
    \item State the criticism. Justify the criticism.
    \item State the criticism. Justify the criticism.
    \item State the criticism. Justify the criticism.
    \item State the criticism. Justify the criticism.
    \item State the criticism. Justify the criticism.
\end{enumerate}

\end{document}
